\documentclass{article}
\usepackage{graphicx} % Required for inserting images
\usepackage[T2A]{fontenc}
\linespread{1.2}
\usepackage{titlesec}
\titlespacing*{\section}{0pt}{10pt}{5pt}
\title{Thermodynamics guide}
\author{Alina Palonskaya}
\date{}
\usepackage[a4paper, left=2cm, right=2cm, top=2cm, bottom=2cm]{geometry}

\begin{document}

\maketitle

\Large\section{Введение}
\large{
Оценочное время освоения- 2 недели

а) Молекулярная физика. Термодинамика. Мякишев Г. Я.

б) Строение и свойства вещества. Бутиков Е.И., Кондратьев А.С.

б*) Физика в задачах: экзаменационные задачи с решениями. Г.Ф.Меледин

\textit{Тема:} Тепловые явления(65 задачи)
}
\Large\section{Основная часть}
\large{
\subsection{Основная теория}
Оценочное время освоения- 3 недели

а) Общий курс физики. Термодинамика и молекулярная физика. Д.В.Сивухин.

а*) Задачи Московских Физических Олимпиад. А.И.Буздин. В.А.Ильин.

\textit{Тема:} Термодинамика(42 задачи)

а**) Раз задача, Два задача... А.И.Буздин. А.Р.Зильберман.

\textit{Тема:} Молекулярная физика и тепловые явления(55 задач)

б) Физика в примерах и задачах. Бутиков Е.И., Быков А.А., Кондратьев А.С. (Термодинамика)

в) Тепловые явления и молекулярная физика. Паршаков.

г*) Физика макросистем. Основные законы. Иродов И.Е.(как дополнение в Сивухину)

\subsection{Основная практика}
Оценочное время освоения- 2 недели

a) Задачи московских городских олимпиад по физике 1986-2005 (79 задач)

б) 200 Интригующих задач по физике(23 задачи)

\textit{Темы:} Поверхностное натяжение, Термодинамика, Фазовые переходы
}
\Large\section{Дополнительно}
\large{
Теорию по темам можно найти в квантах, практику в подборках mathus, на pho.rs. Оценочное время освоения-1 неделя.

а) Вытекание газа из отверстия

б) Расширение в пустоту

в) Распределение по скоростям в различных ск(декартовая, цилиндрическая, сферическая)

г) Энтропия, T-S координаты(X23 Бинарные термодинамические циклы, 200 задач)

д) Испарение, влажность

е) Процесс кипения(квант), пограничное кипение(IZhO 2020, паршаков )

ж) Термоэлектричество(Apho 2018)

з) Условия равновесия при отверстии, сравнимым с длинной пробега/много больше(низкий, средний и высокий вакуум)

и) Неидеальный газ(колличественно поправки)

к) Еще 200 интригующих задач по физике(31 задача)

л) Задачи по физике. И.Ш.Слободецкий. Л.Г.Асламазов.(30 задач)
}


\end{document}
