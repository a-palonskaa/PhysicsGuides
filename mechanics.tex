\documentclass{article}
\usepackage{graphicx} % Required for inserting images
\usepackage[T2A]{fontenc}
\linespread{1.2}
\usepackage{titlesec}
\titlespacing*{\section}{0pt}{10pt}{5pt}
\title{Mechanics guide}
\author{Alina Palonskaya}
\date{}
\usepackage[a4paper, left=2cm, right=2cm, top=2cm, bottom=2cm]{geometry}

\begin{document}

\maketitle
\Large\section{Введение}

\large{Оценочное время освоения- 3 недели

а) Физика. Механика. 10 класс.  Под ред. Мякишева Г.Я.

а*) Физика. Механика. Бутиков Е.И., Кондратьев А.С.

б) Физика в задачах: экзаменационные задачи с решениями. Г.Ф.Меледин

\textit{Тема:} Механика(182 задачи) }

\Large\section{Основная часть}
\subsection{Основная теория}

\large{Оценочное время освоения- 3 недели

а) Основные законы механики. Иродов И.Е.

а*) Задачи Московских Физических Олимпиад. А.И.Буздин. В.А.Ильин.

\textit{Тема:} Механика(116 задач)

б) Д.В. Сивухин. Общий курс физики. Механика. т.I.

б*) Раз задача, Два задача... А.И.Буздин. А.Р.Зильберман.

\textit{Тема:} Механика(86 задач)

\textbf{в)} Механика, Колебания, Акустика. Паршаков.

\textbf{г)} Физика в примерах и задачах. Бутиков Е.И., Быков А.А., Кондратьев А.С. (Механика)

\subsection{Колебания}
Оценочное время освоения- 1 неделя

a) Колебания и волны. М.Н.Перунова.

а*) Уравнение колебаний. 1. Mathus. (выборочно, в качестве практики к книге)

а**) Уравнение колебаний. 2. Mathus.(-//-)

\subsection{Основная практика}
Оценочное время освоения- 4 недели

a) Задачи московских городских олимпиад по физике 1986-2005 (254 задачи)

\textit{Тема:} Механика

б) 200 Интригующих задач по физике(121 задача)

\textit{Темы:} Кинематика, Динамика, Гравитация, Механическая энергия, Соударения, Механика твердого тела, Статика, Веревки и цепи, Гидростатика и динамика

в)Польские Физические Олимпиады. В.Гориковский.(56 задач)

\textit{Темы:} Соударения; Колебания; Полотер; Плоскость, транспортер, винт, шар, вал; Равновесие и устойчивость; Разное(механику)

\Large\section{Дополнительно}
\large{
Оценочное время освоения- 2 недели

а) Цепная линия. Метод виртуальных перемещений.

б) Идея задачи IPhO2014 1A

в) Работа силы трения(квант, задача о трех досках)

г) Горизонтальная сила Архимеда(подборка mathus)

д) Сила Кориолиса, потенциал центробежной силы.

е) Свойства эллипса, движение небесных тел в полярной СК(mathus "Законы Кеплера", квант)

ж) Система ЦМ в задачах на соударения, векторная интерпритация.

з) Еще 200 интригующих задач по физике(131 задача)

и) Задачи по физике. И.Ш.Слободецкий. Л.Г.Асламазов. (69 задач)

r) Матрица лоренца.

}
}


\end{document}
