\documentclass{article}
\usepackage{graphicx} % Required for inserting images
\usepackage[T2A]{fontenc}
\linespread{1.2}
\usepackage{titlesec}
\titlespacing*{\section}{0pt}{10pt}{5pt}
\title{Optics and quantum mechanics guide}
\author{Alina Palonskaya}
\date{}
\usepackage[a4paper, left=2cm, right=2cm, top=2cm, bottom=2cm]{geometry}

\begin{document}

\maketitle
\Large\section{Теоретическая база}
\large{
Особый фокус на визуальную интерпретацию, оптические схемы.

а) Оптика. Квантовая физика. 11класс. Мякишев Г.Я., Синяков А.З.

б) И.В.Савельев. Курс общей физики, том III. Оптика, Атомная физика, Физика атомного ядра и элементарных частиц.

в) Электродинамика. Оптика. Е.И.Бутиков, А.С.Кондратьев.

!Особое внимание на описание оптических приборов и векторные способы расчета интерференционной картины.

г) Волновые процессы. Основные законы. Иродов И.Е.

д) Квантовая физика. Основные законы. Иродов И.Е.

е) Оптика в ключевых задачах. Паршаков А.Н.

ж) Квантовая физика в избранных задачах. Паршаков А.Н.

з) Физика в примерах и задачах. Бутиков Е.И., Быков А.А., Кондратьев А.С.
}
\Large\section{Практика}
\large{
Дополнительно можно искать задачи в Kevin Zhoe handouts, после статей в квантах, на math.us и pho.rs

а) Физика в задачах: экзаменационные задачи с решениями. Г.Ф.Меледин (52 задачи)

б) Задачи Московских Физических Олимпиад. А.И.Буздин. В.А.Ильин. (20 задач)

в) Раз задача, Два задача... А.И.Буздин. А.Р.Зильберман. (26 задач)

г) Задачи московских городских олимпиад по физике 1986-2005 (46 задач)

д) Задачи по общей физике. Иродов. И. Е. (250 оптика+ 245 кванты- решать выборочно по темам)

е) 200 Интригующих задач по физике (9 задач)

ж) Ещё 200 интригующих задач по физике (13 задач)

и) Задачи к Фейнмановским лекциям по физике под редакцией М. А. Готтлиба и Р. Пфайффера

к) Польские Физические Олимпиады. В.Гориковский. (9 задач)
}

\Large\section{Дополнительные темы к проработке}
\large{
Проверить себя на умение расчитывать простейшие интерференционные схемы- если надо, вернуться к Савельеву/Бутикову.

а) Инвариант Аббе.

б) Применение уравнения Шрёдингера для нахождения волновой функции в потенциальной яме, баръере, гармонический осциллятор, туннельный эффект. (Можно посмотреть лабоаторные работы).

в) Статьи кванта по геометрической оптике(геометрические построения).

г) Микроскопы, телескопы, их разрешающая способность.

д) Псевдоэксперимент IZhO 2023(на волновую оптику).

е) Эксперименты с российских сборо(pho.rs, Квалификационные сборы)

ж) Поляризаторы. Четвертьволновые, полуволновые пластинки. Индикатриса рассеяния.
}

\end{document}
