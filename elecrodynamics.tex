\documentclass{article}
\usepackage{graphicx} % Required for inserting images
\usepackage[T2A]{fontenc}
\linespread{1.2}
\usepackage{titlesec}
\titlespacing*{\section}{0pt}{10pt}{5pt}
\title{Electrodynamics guide}
\author{Alina Palonskaya}
\date{}
\usepackage[a4paper, left=2cm, right=2cm, top=2cm, bottom=2cm]{geometry}

\begin{document}

\maketitle

\Large\section{Введение}
\large{
Оценочное время освоения-2 недели

а) Электродинамика. 10-11классы. Мякишев Г.Я., Синяков А.З., Слободсков Б.А.

а*) \textbf{Рисовать магнитные линии различных соленоидов, намагниченных предметов.}

б) Электродинамика. Оптика. Е.И.Бутиков, А.С.Кондратьев.

б*) Физика в задачах: экзаменационные задачи с решениями. Г.Ф.Меледин (128 задач)
}

\Large\section{Основная часть}
\large{
\subsection{Основная теория}
Оценочное время освоения- 4 недели

а) Электричество. Сивухин Д.В.

а*) Задачи Московских Физических Олимпиад. А.И.Буздин. В.А.Ильин. (62 задачи)

б) Основные законы электромагнетизма. Иродов И.Е.

! Разобраться с геометрическим смыслом grad, div, rot, уравнений Максвелла.

б*) Раз задача, Два задача... А.И.Буздин. А.Р.Зильберман. (80 задач)

в) Электромагнетизм в ключевых задачах. Паршаков А.Н.

г)  Физика в примерах и задачах. Бутиков Е.И., Быков А.А., Кондратьев А.С.

д) Берклеевский курс физики. Том II. Электричество и Магнетизм.

\subsection{Основная практика}
Оценочное время освоения- 2 недели

а) Задачи московских городских олимпиад по физике 1986-2005 (96 задач)

б) 200 Интригующих задач по физике(62 задачи)

\textit{Темы:} Электростатика, Магнитостатика, Электрические цепи, Электромагнетизм

в) Польские Физические Олимпиады. В.Гориковский.(25 задач)

\textit{Темы:} Электрические цепи, Электромагнитные явления

г) Задачи по общей физике. Иродов. И. Е. (там 438 задач- из них можно выбрать те темы, по которым меньший процент выполнения по прошлым сборникам)
}

\Large\section{Дополнительно}
\large{
Оценочное время освоения- 2 недели

а) Конденсаторы(статьи квантов+ упражнения)

б) Импеданс, метод комплексных амплитуд, векторные диаграммы(в статьях для радиотехников много примеров)

в) Магнитный и электрический диполь(расчет поля через потенциал, как сумма двух полей, через уравнения максвелла; взаимодействия диполь-заряд, диполь-диполь, диполь-сфера)

г) Эффект Холла

д) Задача о силе взаимодействия двух полусфер.

е) Динамо-машина.

ж) Еще 200 интригующих задач по физике(71 задача)

к) Плейлист "Электромагнитная индукция" youtube канала GetAClass
}

\end{document}
